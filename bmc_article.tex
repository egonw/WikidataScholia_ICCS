%% BioMed_Central_Tex_Template_v1.06
%%                                      %
%  bmc_article.tex            ver: 1.06 %
%                                       %

%%IMPORTANT: do not delete the first line of this template
%%It must be present to enable the BMC Submission system to
%%recognise this template!!

%%% additional documentclass options:
%  [doublespacing]
%  [linenumbers]   - put the line numbers on margins

%%% loading packages, author definitions

%\documentclass[twocolumn]{bmcart}% uncomment this for twocolumn layout and comment line below
%\documentclass[doublespacing]{bmcart}
\documentclass{bmcart}

%%% Load packages
%\usepackage{amsthm,amsmath}
%\RequirePackage{natbib}
%\RequirePackage[authoryear]{natbib}% uncomment this for author-year bibliography
\RequirePackage{hyperref}
\usepackage[utf8]{inputenc} %unicode support

%%% Begin ...
\begin{document}

%%% Start of article front matter
\begin{frontmatter}

\begin{fmbox}
\dochead{Software}

\title{Wikidata and Scholia as a hub linking chemical knowledge}

\author[
   addressref={aff1},
   email={egon.willighagen@maastrichtuniversity.nl}
]{\inits{EW}\fnm{Egon} \snm{Willighagen}}
\author[
   addressref={aff1},
   email={FIXME@maastrichtuniversity.nl}
]{\inits{DS}\fnm{Denise} \snm{Slenter}}
\author[
   addressref={aff2},
   email={FIXME@FIXME.MORE}
]{\inits{DM}\fnm{Daniel} \snm{Mietchen}}
\author[
   addressref={aff1,aff3},
   email={FIXME@FIXME.MORE}
]{\inits{CE}\fnm{Chris} \snm{Evelo}}
\author[
   addressref={aff4},
   email={FIXME@FIXME.MORE}
]{\inits{FN}\fnm{Finn} \snm{Nielsen}}

\address[id=aff1]{
  \orgname{Department of Bioinformatics - BiGCaT, NUTRIM, Maastricht University},
  %\street{P.O. Box 616, UNS 50 Box19},
  %\postcode{NL-6200 MD},
  \city{Maastricht},
  \cny{NL}
}
\address[id=aff2]{%
  \orgname{Data~Science~Institute, University of Virginia},
  %\street{D\"{u}sternbrooker Weg 20},
  %\postcode{24105}
  \city{Charlottesville, Virginia},
  \cny{USA}
}
\address[id=aff3]{%
  \orgname{Maastricht Centre for Systems Biology - MaCSBio, Maastricht University},
  %\street{D\"{u}sternbrooker Weg 20},
  %\postcode{24105}
  \city{Maastricht},
  \cny{NL}
}
\address[id=aff4]{%
  \orgname{Cognitive Systems, DTU Compute, Technical University of Denmark},
  %\street{D\"{u}sternbrooker Weg 20},
  %\postcode{24105}
  %\city{Maastricht},
  \cny{DK}
}

%\begin{artnotes}
%\note{Sample of title note}     % note to the article
%\note[id=n1]{Equal contributor} % note, connected to author
%\end{artnotes}

\end{fmbox}% comment this for two column layout


\begin{abstractbox}

\begin{abstract}
\end{abstract}

\begin{keyword}
\kwd{Wikidata}
\kwd{Scholia}
\kwd{literature}
\kwd{database}
\end{keyword}

\end{abstractbox}

\end{frontmatter}

\section*{Introduction}

Making chemical databases more FAIR (findable, accessible, interoperable, and reusable) benefits
computational chemistry and cheminformatics. We here discuss Wikidata, a young sister project of
Wikipedia, with one key difference: it is a machine readable database, making it far more useful for
interoperability of molecular databases in systems biology~\cite{Mietchen2015,Putman2017}. Thanks to
the WikiProject Chemistry community on Wikidata, there is a growing amount of information about
chemical compounds.

\section*{Implementation}

\subsection*{User Interface: Scholia}

Scholia is a Python/Flask-based server system that creates webpages using a template
approach~\cite{Nielsen2017}.
It defines templates for concepts around knowledge exchange, such as publications, journals,
publishers, but also topics. It uses SPARQL queries against the Wikidata Query Service (WDQS,
\href{http://query.wikidata.org/}{query.wikidata.org}) and visualizes the data in various forms.

\subsection*{Content Generation: Bioclipse}

Furthermore, we used a combination of Bioclipse (\href{http://bioclipse.net/}{bioclipse.net})~\cite{Spjuth2007,Spjuth2009}
and QuickStatements to add missing chemical compounds for biological
pathways from WikiPathways~\cite{Pico2008,Slenter2018}. Where needed, new Wikidata properties were proposed.

\section*{Results}

We here introduce our contributions to the WikiProject Chemistry to support FAIR-ification of open chemical knowledge. For example, we proposed new Wikidata properties to annotate compounds with external database identifiers for the EPA CompTox Dashboard~\cite{Williams2017},
the SPLASH~\cite{Wohlgemuth2016}, and MetaboLights~\cite{Haug2013}. We also introduced a
Scholia extension~\cite{Nielsen2017}, visualizing data about chemicals and chemical classes: \href{https://tools.wmflabs.org/scholia}{tools.wmflabs.org/scholia}.

\section*{Discussion}

\section*{Conclusion}


%\section*{Figures}
%  \begin{figure}[h!]
%  \caption{\csentence{Sample figure title.}
%      A short description of the figure content
%      should go here.}
%      \end{figure}

\begin{backmatter}

\section*{Availability and requirements}

\begin{itemize}
  \item \textbf{Project name}: Scholia
  \item \textbf{Project home page}: https://github.com/fnielsen/scholia
  \item \textbf{Operating system(s)}: Platform independent
  \item \textbf{Programming language}: Python
  \item \textbf{Other requirements}: Flask, Wikidata
  \item \textbf{License}: GPL v3
\end{itemize}

\section*{Competing interests}

\section*{Funding}

\section*{Author's contributions}

\section*{Acknowledgements}

\bibliographystyle{vancouver} % Style BST file (bmc-mathphys, vancouver, spbasic).
\bibliography{bmc_article}      % Bibliography file (usually '*.bib' )


%\section*{Additional Files}
%  \subsection*{Additional file 1 --- Sample additional file title}
%    Additional file descriptions text (including details of how to
%    view the file, if it is in a non-standard format or the file extension).  This might
%    refer to a multi-page table or a figure.
%
%  \subsection*{Additional file 2 --- Sample additional file title}
%    Additional file descriptions text.


\end{backmatter}
\end{document}
